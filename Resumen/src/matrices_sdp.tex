% matrices_sdp

Sea $A \in \mathbb{R}^{n \times n}$, $A$ se dice símetrica definida positiva (sdp) si y sólo si

\begin{itemize}
    \item $A = A^t$, simétrica.
    \item $x^{t}Ax > 0$ para todo $x \in \mathbb{R}^{n},~ x \neq 0$, definida positiva.
\end{itemize}

\subsection{Propiedades}\label{matrices_sdp_propiedades}

\begin{itemize}
    \item $A$ es no singular.
    \item $a_{ii} > 0$ para todo $i = 1,\ldots,n$.
    \item Toda submatriz principal es sdp (por lo tanto no singular).
    \item Existe factorización $LU$.
    \item $A$ sdp $\iff$ $B^{t}AB$ es sdp con $B$ no singular.
    \item La submatriz conformada por las filas $2$ a $n$ y columnas $2$ a $n$ después del primer paso de
    la eliminación gaussiana es sdp.
    \item Se puede realizar el método de eliminación gaussiana sin necesidad de permutar filas.
\end{itemize}

\subsection{Adicionales}\label{matrices_sdp_adicionales}

\noindent Sean $A,B \in \mathbb{R}^{n \times n}$ definidas positivas.

\begin{itemize}
    \item[-] ¿Es $A^t$ definida positiva?

    \[x^{t}A^{t}x = {(Ax)}^{t}{(x^t)}^{t} = {(x^{t}Ax)}^{t} = x^{t}Ax > 0\]

    \item[-] ¿Es $A+B$ definida positiva?

    \[x^{t}(A+B)x = \underbrace{x^{t}Ax}_{ > 0} + \underbrace{x^{t}Bx}_{ > 0} > 0\]

    \item[-] ¿Es $A^2$ definida positiva?

    tengo que $e^{t}_{i}Ae_{j} = a_{ij}$, luego, si tengo 

    \[A^2 = 
    \begin{bmatrix}
        1 & -1 \\
        1 & 1
    \end{bmatrix}
    \cdot
    \begin{bmatrix}
        1 & -1 \\
        1 & 1
    \end{bmatrix}
    =
    \begin{bmatrix}
        0 & -2 \\
        2 & 0
    \end{bmatrix}
    \]

    obtengo que con $x = e_1$

    \[
    \begin{bmatrix}
        1 & 0
    \end{bmatrix}
    \begin{bmatrix}
        0 & -2 \\
        2 & 0
    \end{bmatrix}
    \begin{bmatrix}
        1 \\ 0
    \end{bmatrix}
    =
    \begin{bmatrix}
        1 & 0
    \end{bmatrix}
    \begin{bmatrix}
        0 \\ 2
    \end{bmatrix}
    = 0
    \]    

    \item[-] ¿Es $A^2$ sdp si $A$ es sdp?

    Tengo que $y^{t}y = {||y||}_{2}^{2}$, y $A^{2} = A^{t}A$, luego

    \[x^{t}A^{2}x = x^{t}A^{t}Ax = (x^{t}A^{t})(Ax) = {(Ax)}^{t}(Ax) = {||Ax||}^{2}_{2} > 0\]

    pues $Ax \neq 0$ para todo $x \neq 0$ ($A$ es inversible).

\end{itemize}

\subsection{Enunciados de la guía práctica}\label{subsec:enunciados_guia_3_sdp}

\subsubsection{Ejercicio 4}\label{subsubsec:guia_4_ej_4}

Sea $A$ una matriz simétrica y definida posititiva. Probar que $\forall x, y \in \mathbb{R}^{n}$ se cumple:

\begin{itemize}
    \item[a)] Si $x$ e $y$ son linealmente independientes, $|x^{t}Ay| < \sqrt{x^{t}Ax}\sqrt{y^{t}Ay}$
    \item[b)] Si $x$ e $y$ son linealmente independientes, $|x^{t}Ay| = \sqrt{x^{t}Ax}\sqrt{y^{t}Ay}$
\end{itemize}
