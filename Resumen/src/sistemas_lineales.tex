% sistemas_lineales

Un sistema lineal, se plantea frente a una operación matricial compuesto (por ejemplo) de $A \in \mathbb{R}^{n\times n}$, $b \in \mathbb{R}^{n}$, y se busca $x \in \mathbb{R}^{n}$ tal que $Ax = b$.

\

Donde estos elements son de la forma

\[
A = 
\begin{bmatrix}
a_{11} & a_{12} & \ldots & a_{1n} \\
a_{21} & a_{22} & \ldots & a_{2n} \\
\vdots & \vdots & \ddots & \vdots \\
a_{n1} & a_{n2} & \ldots & a_{nn} \\
\end{bmatrix}
~~~~~~~~~~~~~~
b =
\begin{bmatrix}
b_1 \\
b_2 \\
\vdots \\
b_n 
\end{bmatrix}
~~~~~~~~~~~~~~
x = 
\begin{bmatrix}
x_1 \\
x_2 \\
\vdots \\
x_n
\end{bmatrix}
\]

y se plantea el sistema de ecuaciones

\[
\begin{matrix}
a_{11}x_{1} + & a_{12}x_{2} + & \ldots & a_{1n}x_n & = b_1 \\
a_{21}x_{1} + & a_{22}x_{2} + & \ldots & a_{2n}x_n & = b_2 \\
\vdots & \vdots & \ddots & \vdots & \vdots \\
a_{n1}x_{1} + & a_{n2}x_{2} + & \ldots & a_{nn}x_n & = b_n
\end{matrix}
\]

\subsection{Soluciones y Sistemas equivalentes}
\label{subsec:soluciones_y_sistemas_equivalentes}


\begin{itemize}
    \item Si $rang(A) = n \implies$ existe un única solución
    \item Si $rang(A) < n$:
    \begin{itemize}
        \item Si $b \notin Im(A) \implies$ el sistema no tiene solución.
        \item Si $b \in Im(A) \implies$, el sistema tiene infinitas soluciones. 
    \end{itemize}
\end{itemize}

Sean $A \in \mathbb{R}^{n \times n}$, $b \in \mathbb{R}^{n}$, $B \in \mathbb{R}^{n \times n}$ y $d \in \mathbb{R}^{n}$

\

Los sistemas $Ax = b$ y $Bx = d$ son equivalentes si tienen el mismo conjunto de soluciones. Esto se puede aprovechar transformando la matriz original $A$ a una matriz asociada $B$ con el mismo conjunto de soluciones que sea más fácil de resolver.

\subsection{Sistemas de ecuaciones fáciles} 
\label{subsec:sistemas_de_ecuaciones_faciles}

\subsubsection{Matriz diagonal}
\label{subsubsec:matriz_diagonal}

$A = D \in \mathbb{R}^{n \times n}$ matriz diagonal, $b \in \mathbb{R}^{n}$. Es dicha situación se tiene un sistema de la forma

\[
\begin{matrix}
a_{11}x_{1} &  &  &  &  & = b_1 \\
 & \ddots &  & &  & \vdots \\
 & & a_{ii}x_{i} & & & = b_{i} \\
 & &  & \ddots & & \vdots \\
 & & & & a_{nn}x_n & = b_n
\end{matrix}
\]

\begin{itemize}
    \item Si $a_{ii} \neq 0 ~\forall~ i \in \{1,\ldots,n\}$, existe solución y es única. En dicho caso, el algoritmo tiene la forma
    
    \
    
    \begin{algorithm}
    \caption{Simple substitution}
    \label{alg:simple_substitution}
    \begin{algorithmic}
    \For{$i \in [1,\ldots,n]$}
        \State $x_i \gets b_i/a_{ii}$    
    \EndFor
    \end{algorithmic}
    \end{algorithm}
    
    El mismo tiene una complejidad que pertenece a $O(n)$.
    
    \item Si existe $a_{ii} = 0$
    \begin{itemize}
        \item Si $b_{i} = 0$, $x_i$ puede tomar cualquier valor.
        \item Si $b_{i} \neq 0$. no existe solución.
    \end{itemize}
    
\end{itemize}

\subsubsection{Matriz triangular superior}
\label{subsubsec:matriz_triangular_superior}

$A = U \in \mathbb{R}^{n \times n}$ matriz triangular superior, $b \in \mathbb{R}^{n}$. Es dicha situación se tiene un sistema de la forma

\[
\begin{matrix}
a_{11}x_{1} & \ldots & a_{1i}x_i & \ldots & a_{1n}x_n & = b_1 \\
 & \ddots & \vdots & \ddots & \vdots & \vdots \\
 & & a_{ii}x_{i} & \ldots & a_{in}x_n & = b_{i} \\
 & &  & \ddots & \vdots & \vdots \\
 & & & & a_{nn}x_n & = b_n
\end{matrix}
\]

\begin{itemize}
    \item Si $a_{ii} \neq 0 ~\forall~ i \in \{1,\ldots,n\}$, existe solución y es única. En dicho caso, el algoritmo tiene la forma
    
    \
    \begin{algorithm}
    \begin{algorithmic}
    \caption{Backward substitution}
    \label{alg:backward_substitution}
    \State $x_n \gets b_n/a_nn $
    \For{$i \in [n,\ldots,2]$}
        \State $x_i \gets (b_i - \sum_{j=i+1}^{n} a_{ij}x_j)/a_{ii}$
    \EndFor
    \end{algorithmic}
    \end{algorithm}
    
    El mismo tiene una complejidad que pertenece a $O(n^2)$.
    
    \item Si existe $a_{ii} = 0$
    \begin{itemize}
        \item Si $b_{i} - \sum_{j=i+1}^{n} a_{ij}x_j = 0$, $x_i$ puede tomar cualquier valor.
        \item Si $b_{i} - \sum_{j=i+1}^{n} a_{ij}x_j \neq 0$. no existe solución.
    \end{itemize}
\end{itemize}

\subsubsection{Matriz triangular inferior}
\label{subsubsec:matriz_triangular_inferior}

$A = L \in \mathbb{R}^{n \times n}$ matriz triangular superior, $b \in \mathbb{R}^{n}$. Es dicha situación se tiene un sistema de la forma

\[
\begin{matrix}
a_{11}x_{1} &  &  &  &   & = b_1 \\
\vdots &  \ddots & &  &  & \vdots \\
a_{i1}x_{1} & \ldots & a_{ii}x_{i} & & & = b_{i} \\
\vdots & \ddots & \vdots & \ddots& & \vdots \\
a_{n1}x_{1} & \ldots &  a_{ni}x_i & \ldots & a_{nn}x_n & = b_n
\end{matrix}
\]

\begin{itemize}
    \item Si $a_{ii} \neq 0 ~\forall~ i \in \{1,\ldots,n\}$, existe solución y es única. En dicho caso, el algoritmo tiene la forma
    
    \
    \begin{algorithm}
    \begin{algorithmic}
    \caption{Forward substitution}
    \label{alg:forward_substitution}
    \State $x_1 \gets b_1/a_11 $
    \For{$i \in [2,\ldots,n]$}
        \State $x_i \gets (b_i - \sum_{j=1}^{i-1} a_{ij}x_j)/a_{ii}$
    \EndFor
    \end{algorithmic}
    \end{algorithm}
    
    El mismo tiene una complejidad que pertenece a $O(n^2)$.
    
    \item Si existe $a_{ii} = 0$
    \begin{itemize}
        \item Si $b_i - \sum_{j=1}^{i-1} a_{ij}x_j = 0$, $x_i$ puede tomar cualquier valor.
        \item Si $b_i - \sum_{j=1}^{i-1} a_{ij}x_j \neq 0$. no existe solución.
    \end{itemize}
\end{itemize}
