% factorizacion_lu

Resolver varios sistemas de ecuaciones con Eliminación Gaussiana tiene un costo que pertenece a $O(n^3)$ por cada uno. Hay una manera de evitar esto con una factorización llamada LU.

Sean $A \in \mathbb{R}^{n \times n}$, $L \in \mathbb{R}^{n \times n}$ triangular inferior con unos en su diagonal principal, $U \in \mathbb{R}^{n \times n}$ triangular superior, y por último 

\[A = LU\]

\subsection{Utilidad}
\label{subsec:utilidad_lu}

Se tiene que $Ax = b$, luego

\[LUx = b\]

y se resuelve en dos etapas:

\[Ly = b\]
\[Ux = y\]

los cuales son dos sistemas triangulares con un costo de computo que pertenece a $O(n^2)$.

\subsection{Matriz de transformación gaussiana}
\label{subsec:matriz_de_transformacion_gaussiana}

Sea $A \in \mathbb{R}^{n \times n}$. Supongamos que aplicamos eliminación Gaussiana y se verifica que $a_{ii}^{i - 1}  \neq 0$ para todo $i = 1,\ldots,n-1$.

\

Luego, sea la matriz elemental (tipo2), y $m_{ji} = \frac{a_{ji}^{i-1}}{a_{ii}^{i-1}}$

\[
E = 
\begin{bmatrix}
1 & 0 & 0 & \ldots & 0 \\
-m_{21} & 1 & 0 & \ldots & 0 \\
0 & 0 & 1 & \ldots & 0 \\
\vdots & \vdots & \vdots & \ddots & \vdots \\
0 & 0 & 0 & \ldots & 1 \\
\end{bmatrix}
\]

y con ello puedo realizar:

\[
\begin{bmatrix}
1 & 0 & 0 & \ldots & 0 \\
-m_{21} & 1 & 0 & \ldots & 0 \\
0 & 0 & 1 & \ldots & 0 \\
\vdots & \vdots & \vdots & \ddots & \vdots \\
0 & 0 & 0 & \ldots & 1 \\
\end{bmatrix}
\begin{bmatrix}
a_{11}^{0} & a_{12}^{0} & \ldots & a_{1j}^{0} & \ldots &a_{1n}^{0} \\
a_{21}^{0} & a_{22}^{0} & \ldots & a_{2j}^{0} & \ldots &a_{2n}^{0} \\
a_{31}^{0} & a_{32}^{0} & \ldots & a_{3j}^{0} & \ldots &a_{3n}^{0} \\
\vdots & \vdots & \ddots & \vdots & \ddots & \vdots \\
a_{n1}^{0} & a_{n2}^{0} & \ldots & a_{nj}^{0} & \ldots &a_{nn}^{0}
\end{bmatrix}
=
\]
\[
\begin{bmatrix}
a_{11}^{0} & a_{12}^{0} & \ldots & a_{1j}^{0} & \ldots &a_{1n}^{0} \\
a_{21}^{0} -m_{21}a_{11}^{0} & a_{22}^{0}-m_{21}a_{12}^{0} & \ldots & a_{2j}^{0}-m_{21}a_{1j}^{0} & \ldots &a_{2n}^{0}-m_{21}a_{1n}^{0} \\
a_{31}^{0} & a_{32}^{0} & \ldots & a_{3j}^{0} & \ldots &a_{3n}^{0} \\
\vdots & \vdots & \ddots & \vdots & \ddots & \vdots \\
a_{n1}^{0} & a_{n2}^{0} & \ldots & a_{nj}^{0} & \ldots &a_{nn}^{0}
\end{bmatrix}
\]

\

lo cual es lo mismo que la operación $F_2 - m_{21}F_1$ de la Eliminación Gaussiana. Con esto se puede pensar en una matriz llamada \emph{primera matriz de la eliminación gaussiana} de la forma

\[
M^1 =
\begin{bmatrix}
1 & 0 & 0 & \ldots & 0 & \ldots & 0 \\
-m_{21} & 1 & 0 & \ldots & 0 & \ldots & 0 \\
-m_{31} & 0 & 1 & \ldots & 0 & \ldots & 0 \\
\vdots & \vdots & \vdots & \ddots & \vdots & \ddots & \vdots \\
-m_{i1} & 0 & 0 & \ldots & 1 & \ldots & 0 \\
\vdots & \vdots & \vdots & \ddots & \vdots & \ddots & \vdots \\
-m_{n1} & 0 & 0 & \ldots & 0 & \ldots & 1 
\end{bmatrix}
\]

y con la misma puedo realizar el producto 

\[
\begin{bmatrix}
1 & 0 & 0 & \ldots & 0 & \ldots & 0 \\
-m_{21} & 1 & 0 & \ldots & 0 & \ldots & 0 \\
-m_{31} & 0 & 1 & \ldots & 0 & \ldots & 0 \\
\vdots & \vdots & \vdots & \ddots & \vdots & \ddots & \vdots \\
-m_{i1} & 0 & 0 & \ldots & 1 & \ldots & 0 \\
\vdots & \vdots & \vdots & \ddots & \vdots & \ddots & \vdots \\
-m_{n1} & 0 & 0 & \ldots & 0 & \ldots & 1 
\end{bmatrix}
\begin{bmatrix}
a_{11}^{0} & a_{12}^{0} & \ldots & a_{1j}^{0} & \ldots &a_{1n}^{0} \\
a_{21}^{0} & a_{22}^{0} & \ldots & a_{2j}^{0} & \ldots &a_{2n}^{0} \\
a_{31}^{0} & a_{32}^{0} & \ldots & a_{3j}^{0} & \ldots &a_{3n}^{0} \\
\vdots & \vdots & \ddots & \vdots & \ddots & \vdots \\
a_{i1}^{0} & a_{i2}^{0} & \ldots & a_{ij}^{0} & \ldots &a_{in}^{0} \\
\vdots & \vdots & \ddots & \vdots & \ddots & \vdots \\
a_{n1}^{0} & a_{n2}^{0} & \ldots & a_{nj}^{0} & \ldots &a_{nn}^{0}
\end{bmatrix}
=
\]
\[
\begin{bmatrix}
a_{11}^{0} & a_{12}^{0} & \ldots & a_{1j}^{0} & \ldots &a_{1n}^{0} \\
a_{21}^{0} -m_{21}a_{11}^{0} & a_{22}^{0}-m_{21}a_{12}^{0} & \ldots & a_{2j}^{0}-m_{21}a_{1j}^{0} & \ldots &a_{2n}^{0}-m_{21}a_{1n}^{0} \\
a_{31}^{0} -m_{31}a_{11}^{0} & a_{32}^{0}-m_{31}a_{12}^{0} & \ldots & a_{3j}^{0}-m_{31}a_{1j}^{0} & \ldots &a_{3n}^{0}-m_{31}a_{1n}^{0} \\
\vdots & \vdots & \ddots & \vdots & \ddots & \vdots \\
a_{i1}^{0} -m_{i1}a_{11}^{0} & a_{i2}^{0}-m_{i1}a_{12}^{0} & \ldots & a_{ij}^{0}-m_{i1}a_{1j}^{0} & \ldots &a_{in}^{0}-m_{i1}a_{1n}^{0} \\
\vdots & \vdots & \ddots & \vdots & \ddots & \vdots \\
a_{n1}^{0} -m_{n1}a_{11}^{0} & a_{n2}^{0}-m_{n1}a_{12}^{0} & \ldots & a_{nj}^{0}-m_{n1}a_{1j}^{0} & \ldots &a_{nn}^{0}-m_{n1}a_{1n}^{0} \\
\end{bmatrix}
\]

\

Lo cual no es más que el primer paso de la Eliminación Gaussiana. De la misma forma, se tiene la i-ésima matriz de la transformación gaussiana para el i-ésimo paso de la eliminación gaussiana 

\[
M^i =
\begin{bmatrix}
1 & \ldots & 0  & 0 & \ldots & 0 \\
\vdots & \ddots & \vdots & \vdots & \ddots & 0 \\
0  & \ldots & 1 & 0 & \ldots & 0 \\
0 & \ldots & -m_{i+1i} & 1 & \ldots & 0 \\
\vdots  & \ddots & \vdots & \vdots & \ddots & \vdots \\
0 & \ldots & -m_{ni} & 0 & \ldots & 1 
\end{bmatrix}
\]

y se pude realizar el i-ésimo paso de la eliminación gaussiana.

\[
\begin{bmatrix}
1 & \ldots & 0  & 0 & \ldots & 0 \\
\vdots & \ddots & \vdots & \vdots & \ddots & 0 \\
0  & \ldots & 1 & 0 & \ldots & 0 \\
0 & \ldots & -m_{i+1i} & 1 & \ldots & 0 \\
\vdots  & \ddots & \vdots & \vdots & \ddots & \vdots \\
0 & \ldots & -m_{ni} & 0 & \ldots & 1 
\end{bmatrix}
\begin{bmatrix}
a_{11}^{i-1} & \ldots & a_{1i}^{i-1} & \ldots & a_{1n}^{i-1} \\
\vdots & \ddots & \vdots & \ddots & \vdots \\
0 & \ldots & a_{ii}^{i-1} & \ldots & a_{in}^{i-1} \\
0 & \ldots & a_{i+1i}^{i-1} & \ldots & a_{i+1n}^{i-1} \\
\vdots & \ddots & \vdots & \ddots & \vdots \\
0 & \ldots & a_{ni}^{i-1} & \ldots & a_{nn}^{i-1} \\
\end{bmatrix}
=
\]
\[
\begin{bmatrix}
a_{11}^{i-1} & a_{12}^{i-1} & \ldots & a_{1i}^{i-1} & \ldots & a_{1n}^{i-1} \\
0 & a_{22}^{i-1} & \ldots & a_{2i}^{i-1} & \ldots & a_{2n}^{i-1} \\
\vdots & \vdots & \ddots & \vdots & \ddots & \vdots \\
0 & 0 & \ldots & a_{ii}^{i-1} & \ldots & a_{in}^{i-1} \\
0 & 0 & \ldots & a_{i+1i}^{i-1}-m_{i+1i}a_{ii}^{i-1} & \ldots & a_{i+1n}^{i-1}-m_{i+1i}a_{in}^{i-1} \\
\vdots & \vdots & \ddots & \vdots & \ddots & \vdots \\
0 & 0 & \ldots & a_{ni}^{i-1}-m_{ni}a_{ii}^{i-1} & \ldots & a_{nn}^{i-1}-m_{ni}a_{in}^{i-1}
\end{bmatrix}
\]

\subsection{Método de Eliminación Gaussiana}
\label{subsec:metodo_de_eliminacion_gaussiana}

Con lo visto en la sección previa, se puede observar que el producto de las distintas $M^i$ con A, realizan el algoritmo de Eliminación Gaussiana visto en la sección \ref{subsec:algotimo_de_eliminacion_gaussiana}, obteniéndose una matriz triangular superior

\[
M^{n-1}M^{n-2}\cdots M^{1}A = U ~\text{con $U$ triangular superior}
\]

\

Un dato \emph{importante} de todo esto es que se asume que $a_{ii}^{i-1} \neq 0$ para todo $i = 1,\ldots,n$.

\subsection{Propiedades de la matriz de transformación gaussiana}

Tengo que puedo ver a $M^i$ como

\[
M^i =
\begin{bmatrix}
1 & \ldots & 0  & 0 & \ldots & 0 \\
\vdots & \ddots & \vdots & \vdots & \ddots & 0 \\
0  & \ldots & 1 & 0 & \ldots & 0 \\
0 & \ldots & -m_{i+1i} & 1 & \ldots & 0 \\
\vdots  & \ddots & \vdots & \vdots & \ddots & \vdots \\
0 & \ldots & -m_{ni} & 0 & \ldots & 1 
\end{bmatrix}
=
I -
\begin{bmatrix}
0 \\
\vdots \\
m_{i+1i} \\
\vdots \\
m_{ni}
\end{bmatrix}
\begin{bmatrix}
0 & \ldots & 1 & \ldots & 0
\end{bmatrix}
\]

es decir 

\[M^i = I - m_{i}^{t}e_i\]

con $m_i = (0, \ldots, m_{i+1i},\ldots,m_{ni})$ y $e_i$ el i-ésimo vector canónico.

\noindent Y se tienen las siguientes propiedades:

\begin{itemize}
    \item $M^i$ es triangular inferior.
    \item $M^i$ es inversible.
\end{itemize}

Veamos un poco esta segunda propiedad

\[(I - m_{i}^{t}e_i)(I + m_{i}^{t}e_i) = I + m_{i}^{t}e_i - m_{i}^{t}e_i - m_{i}^{t}e_im_{i}^{t}e_i = I - m_{i}^{t}e_im_{i}^{t}e_i\]

\[\text{pero}~ e_{i}m_{i}^{t} = 
\begin{bmatrix}
0 & \ldots & 1 & \ldots & 0
\end{bmatrix}
\begin{bmatrix}
0 \\
\vdots \\
m_{i+1i} \\
\vdots \\
m_{ni}
\end{bmatrix}
= 0
\]

\

Entonces tengo que $(I - m_{i}^{t}e_i)(I + m_{i}^{t}e_i) = I \implies {(M^{i})}^{-1} = {(I - m_{i}^{t}e_i)}^{-1} = I + m_{i}^{t}e_i$.

y observando esto se tiene que

\[M^{n-1}M^{n-2}\cdots M^{1}A = U\]
\[A = {(M^1)}^{-1}{(M^2)}^{-1}\cdots{(M^{n-1})}^{-1}U\]
\[A=(I+m_{1}^{t}e_1)(I+m_{2}^{t}e_2)\cdots(I + m_{n-1}^{t}e_{n-1})U\]

y se puede observar que $m_{i}^{t}e_{i}m_{j}^{t}e_{j} = 0$ si $i < j$, con esto

\[A = (I + m_{1}^{t}e_{1} + m_{2}^{t}e_{2} + \cdots + m_{n-1}^{t}e_{n-1})U = LU\]

luego

\[
\begin{bmatrix}
a_{11} & a_{12} & \ldots & a_{1i} & \ldots & a_{1n} \\
a_{21} & a_{22} & \ldots & a_{2i} & \ldots & a_{2n} \\
\vdots & \vdots & \ddots & \vdots & \ddots & \vdots \\
a_{i1} & a_{i2} & \ldots & a_{ii} & \ldots & a_{in} \\
\vdots & \vdots & \ddots & \vdots & \ddots & \vdots \\
a_{n1} & a_{n2} & \ldots & a_{ni} & \ldots & a_{nn}
\end{bmatrix}
=
\begin{bmatrix}
1 & 0 & \ldots & 0 & \ldots & 0 \\
m_{21} & 1 & \ldots & 0 & \ldots & 0 \\
\vdots & \vdots & \ddots & \vdots & \ddots & \vdots \\
m_{i1} & m_{i2} & \ldots & 1 & \ldots & 0 \\
\vdots & \vdots & \ddots & \vdots & \ddots & \vdots \\
m_{n1} & m_{n2} & \ldots & m_{ni} & \ldots & 1
\end{bmatrix}
\begin{bmatrix}
u_{11} & u_{12} & \ldots & u_{1i} & \ldots & u_{1n} \\
0 & u_{22} & \ldots & u_{2i} & \ldots & u_{2n} \\
\vdots & \vdots & \ddots & \vdots & \ddots & \vdots \\
0 & 0 & \ldots & a_{ii} & \ldots & u_{in} \\
\vdots & \vdots & \ddots & \vdots & \ddots & \vdots \\
0 & 0 & \ldots & 0 & \ldots & u_{nn}
\end{bmatrix}
\]

\subsection{Propiedades de LU}
\label{subsec:propiedades_de_lu}

\begin{itemize}
    \item La factorización LU está asociada a la Eliminación Gaussiana sin necesidad de intercambio de filas. Se verifica que $l_{ii} = 1$ para todo $i = 1,\ldots,n$.
    \item No toda matriz tiene factorización LU, por ejemplo 
    $\begin{bmatrix}
    0 & 1 \\
    1 & 0
    \end{bmatrix}$
    \item Si $A \in \mathbb{R}^{n \times n}$ tiene todas sus submatrices principales no singulares, entonces tiene factorización LU.  \emph{Idea} de demostración por inducción:
    \begin{itemize}
        \item Caso Base: Si $n=2$ tengo 
        \[A_2 = 
        \begin{bmatrix}
            a_{11} & a_{12} \\
            a_{21} & a_{22}
        \end{bmatrix}
        \]
        \
        si se puede realizar Eliminación Gaussiana, es porque $a_{11} \neq 0$. Entonces es válido.
        \item Caso inductivo: La \emph{hipótesis inductiva} que se toma es que vale para $A_{n}$, siendo la misma:
        \[
        A_{n} =
        \begin{bmatrix}
            a_{11} & \ldots & a_{1n} \\
            \vdots & \ddots & \vdots \\
            a_{n1} & \ldots & a_{nn}
        \end{bmatrix}
        \]
        
        \        
        
        Luego, quiero ver que vale que para $A_{n+1}$. En particular, si toma la submatriz principal de orden $n$, cumple con todas las hipótesis necesarias para poder afirmar que 
        \[A_{n} = L_{n}U_{n},\]
        y ahora quiero ver que 
        \[A_{n+1} = L_{n+1}U_{n+1},\]
        donde tengo que $L_{n+1}$ y $U_{n+1}$ son de la forma
        \[
        L_{n+1} =
        \begin{bmatrix}
            L_n & 0 \\
            l_{n+1}^{t} & 1
        \end{bmatrix}
        ~~~~~~~~~~~~~~~~~~
        U_{n+1} =
        \begin{bmatrix}
            U_{n} & u_{n+1} \\
            0^t & u_{n+1n+1}
        \end{bmatrix}
        \]
        para averiguar esto necesito analizar la siguiente igualdad
        \[
        A_{n+1} = 
        \begin{bmatrix}
            A_n & c_{n+1} \\
            f_{n+1}^{t} & a_{n+1n+1}
        \end{bmatrix}
        ~
        \overset{?}{=}
        ~
        \begin{bmatrix}
            L_n & 0 \\
            l_{n+1}^{t} & 1
        \end{bmatrix}
        \begin{bmatrix}
            U_{n} & u_{n+1} \\
            0^t & u_{n+1n+1}
        \end{bmatrix}
        =
        \begin{bmatrix}
            L_{n}U_{n} & L_{n}u_{n+1} \\
            l_{n+1}^{t}U_{n} & l_{n+1}^{t}u_{n+1} + u_{n+1n+1}
        \end{bmatrix}
        \]
        y lo único que falta para que esto se cumpla es que
        \begin{itemize}
            \item[-] $c_{n+1} = L_{n}u_{n+1}$
            \item[-] $f_{n+1}^{t} = l_{n+1}^{t}U_{n}$
            \item[-] $a_{n+1n+1} = l_{n+1}^{t}u_{n+1} + u_{n+1n+1}$
        \end{itemize}
        
    \end{itemize}
    \item Si $A \in \mathbb{R}^{n \times n}$ es no singular y existe factorización LU, entonces es única.
    \item Si $A$ es inversible con factorización LU, entonces se puede aplicar Eliminación Gaussiana sin pivoteo.
    \item Si $A \in \mathbb{R}^{n \times n}$ es estrictamente diagonal dominante, entonces tiene factorización LU. \emph{Idea} de demostración: 
    
    \
    
    Para esto se puede probar que si $A$ es una matriz estrictamente diagonal dominante, entonces es inversible. Esto lo pruebo por el absurdo, supongo que $A$ no es inversible, es decir, existe $x^{*} \neq 0$ tal que $Ax^{*} = 0$ (con esto sus filas son linealmente dependientes).
    
    \[\text{como $x^{*} \neq 0$,}~ \exists x_{k}^{*} \in x ~:~ |x_{k}^{*}| = \underset{1 \leq i \leq n}{max}|x_{i}^{*}| > 0\]
    y hasta acá tengo que
    \[\begin{bmatrix}
    a_{k1}, & a_{k2}, & \ldots, & a_{kn}    
    \end{bmatrix}
    \begin{bmatrix}
        x_{1}^{*} \\
        x_{2}^{*} \\
        \vdots \\
        x_{n}^{*}
    \end{bmatrix}
    = 
    \sum_{j =1}^{n} a_{kj}x_{j}^{*} = 0
    \]
    lo cual es lo mismo que
    \[a_{kk}x_{k}^{*} + \sum_{j=1, j\neq k}^{n}a_{kj}x_{j}^{*} = 0\] 
    \[a_{kk}x_{k}^{*} = - \sum_{j=1, j\neq k}^{n}a_{kj}x_{j}^{*}\] 
    y aplicando módulo a ambos lados
    \[|a_{kk}||x_{k}^{*}| = |\sum_{j=1, j\neq k}^{n}a_{kj}x_{j}^{*}| \leq \sum_{j=1, j\neq k}^{n}|a_{kj}||x_{j}^{*}|\]
    y con esto, como $x_{kk}^{*} \neq 0$ y $\frac{|x_{j}^{*}|}{|x_{k}^{*}|}\leq 1$
    \[|a_{kk}| \leq \sum_{j=1, j\neq k}^{n}|a_{kj}|\frac{|x_{j}^{*}|}{|x_{k}^{*}|} \leq \sum_{j=1, j \neq k}^{n}|a_{kj}|\] 
    pero esto resulta absurdo,pues en dicho caso, $A$ no sería estrictamente diagonal dominante.
    
    \
    
    Luego, como todas las matrices principales de $A$ son estrictamente diagonal dominantes, son inversibles, y entonces $A$ tiene factorización $LU$.
    
    \item Toda matriz estrictamente diagonal dominante tiene todas sus submatrices principales diagonal dominantes.
    
    \item Si a una matriz $A$ estrictamente diagonal dominante se le aplica el primer paso de la Eliminación Gaussiana, su submatriz principal resultante también es estrictamente diagonal dominante. \emph{Idea} de demostración:
    
    Hay que tener en cuenta que los coeficientes cambia de la forma 
    
    \[a_{ij}^{1} = a_{ij}^{0} - \frac{a_{i1}^{0}}{a_{11}^{0}}\cdot a_{1j}^{0}\]
    
    y quiero ver que 
    \[\sum_{j = 2, j \neq i}^{n} |a_{ij}^{1}| < |a_{ii}^{1}|\]
    
    procedo a probarlo
    
    \[\sum_{j = 2, j \neq i}^{n} |a_{ij}^{1}| =  \sum_{j=2,j \neq i}^{n}|a_{ij}^{0} - \frac{a_{i1}^{0}}{a_{11}^{0}}\cdot a_{1j}^{0}| \leq \sum_{j=2,j \neq i}^{n}|a_{ij}^{0}| + \sum_{j=2,j \neq i}^{n} |\frac{a_{i1}^{0}}{a_{11}^{0}}\cdot a_{1j}^{0}| < |a_{ii}^{0}|-|a_{i1}^{0}| + \frac{|a_{i1}^{0}|}{|a_{11}^{0}|}\cdot\sum_{j=2, j \neq i}^{n}|a_{1j}^{0}|\]
    y luego se tiene 
    \[< |a_{ii}^{0}|-|a_{i1}^{0}| + \frac{|a_{i1}^{0}|}{|a_{11}^{0}|}\cdot (|a_{11}^{0}| - |a_{1i}^{0}|) = |a_{ii}^{0}| - \frac{|a_{i1}^{0}|}{|a_{11}^{0}|}\cdot |a_{1i}^{0}| \leq |a_{ii}^{0} - \frac{|a_{i1}^{0}|}{|a_{11}^{0}|}\cdot |a_{1i}^{0}| = |a_{ii}^{1}|\]
    y con esto se concluye que
    \[\sum_{j = 2, j \neq i}^{n} |a_{ij}^{1}| < |a_{ii}^{1}|\]
    \item Si $A$ es estrictamente diagonal dominante, su factorización LU es única.
\end{itemize}

\subsection{Factorización PLU}
\label{subsec:factorizacion_plu}

En el caso de que exista $i = 1,\ldots,n$ para el cual $a_{ii}^{i-1} = 0$, se puede continuar la Eliminación Gaussiana mediante permutación de filas y obtener una factorización LU de la matriz original permutada, de la forma 

\[PA = LU\]

siendo $P$ la matriz de permutación que establece como se intercambiaron las filas.

\subsection{Enunciados de la guía práctica}
\label{subsec:enunciados_guia_2}

\subsubsection{Ejercicio 5}
\label{subsubsec:guia_2_ej_5}

Sean $A_1,\ldots,A_k \in \mathbb{R}^{n \times n}$ tales que $A_h = L_h U$ es una factorización $LU$ de $A_h~\forall h$ ($U$ es la misma para todas). Sea $A = \sum_{h=1}^{k}A_h$, probar:
\begin{itemize}
    \item[a.] $A$ tiene factorización $LU$.
    
    Tengo que 
    
    \[A = \sum_{h=1}^{k}A_h = \sum_{h=1}^{k}(L_h U) = (\sum_{h=1}^{k}L_h)U = \underbrace{\frac{1}{k}\cdot(\sum_{h=1}^{k}L_h)}_{\widetilde{L}}\overbrace{k\cdot U}^{\widetilde{U}}\]
    
    Luego, como $\widetilde{L}$ es triangular inferior con unos en su diagonal principal y $\widetilde{U}$ es triangular superior, $A$ tiene factorización $LU$, $A = \widetilde{L}\widetilde{U}$ 
    
    \item[b.] Para $1 \leq i,j, \leq n$, los $m_{ij}$ de la triangulación gaussiana de $A$ son el promedio de los $m_{ij}^{h}$
    
    Tengo que cada $L_h$ es de la forma 
    
    \[
    L_h =
    \begin{bmatrix}
        1 & 0 & \ldots & \ldots & 0 \\
        m_{21}^{h} & 1 & \ddots &  & \vdots \\
        \vdots & \ddots & \ddots & \ddots & \vdots \\
        \vdots &  & \ddots & 1 & 0 \\
        m_{n1}^{h} & \ldots & \ldots & m_{nn-1}^{h} & 1 \\
    \end{bmatrix}
    \]
    
    y tengo que $\widetilde{L}$ es de la forma 
    
    \[
    \widetilde{L} ~=~ \frac{1}{k}\sum_{h = 1}^{k}L_h ~=~
    \begin{bmatrix}
        1 & 0 & \ldots & \ldots & 0 \\
        \frac{\sum_{h=1}^{k}m_{21}^{h}}{k} & 1 & \ddots &  & \vdots \\
        \vdots & \ddots & \ddots & \ddots & \vdots \\
        \vdots &  & \ddots & 1 & 0 \\
        \frac{\sum_{h=1}^{k}m_{n1}^{h}}{k} & \ldots & \ldots & \frac{\sum_{h=1}^{k}m_{nn-1}^{h}}{k} & 1 \\
    \end{bmatrix}
    \]
    
    donde cada $m_{ij} = \frac{\sum_{h=1}^{k}m_{ij}^{h}}{k}$
\end{itemize}

\subsubsection{Ejercicio 9}
\label{subsubsec:guia_2_ej_9}

Sea $A \in \mathbb{R}^{n \times n}$ inversible, $A = TS$, con $T$ triangular inferior y $S$ triangular superior. Probar:

\begin{itemize}
    \item[a.] $T$ y $S$ son inversibles.
    
    Como $A$ es inversible, se tiene que $det(A) \neq 0$, luego
    
    \[0 \neq det(A) = det(TS) = det(T)det(S) \implies det(T) \neq 0 \land det(S) \neq 0\]
    
    lo cual implica que $T$ es inversible y $S$ es inversible.
    
    \item[b.] $A$ tiene $LU$.
    
    Si tomo (por ejemplo)
    
    \[
    D_1 = 
    \begin{bmatrix}
        \frac{1}{2} & 0 \\
        0 & \frac{1}{4}
    \end{bmatrix}
    ~~~~~~~~~~~~~~~~
    D_2 = 
    \begin{bmatrix}
        2 & 0 \\
        0 & 4
    \end{bmatrix}
    \]
    
    tengo que $D_1 D_2 = I$, y luego
    
    \[A = TS = TIS = T(D_1 D_2)S = (TD_1)(D_2 S).\]
    
    con esto, sean $D_1 , D_2 \in \mathbb{R}^{n \times n}$ diagonales, con ${(D_1)}_{ii} =  \frac{1}{T_{ii}}$ y con ${(D_2)}_{ii} = T_{ii}$. Donde $D_1 D_2 = I$.
    
    don lo que $A = TS = TIS = \underbrace{TD_1}_{L}\underbrace{D_2 S}_{U}$. Luego, quiero ver que esto es una factorización $LU$.
    
    Tengo que $L$ es triangular inferior con unos en su diagonal y $U$ triangular superior. Adicionalmente
    
    \[(L)_{ii} = \underbrace{fila_i(T)}_{(*,T_{ii},0)}\cdot \underbrace{columna_i(D_1)}_{(0,\frac{1}{T_{ii}},0)} = 1 ~~\forall i = 1,\ldots,n\]
    
    con lo que se encontró $L = TD_1$ triangular inferior con unos en la diagonal y $U = D_2 S$ triangular superior, de forma que $A = LU$.
    
    \item[c.] La matriz $\begin{bmatrix}
        A & b \\
        c^t & d
    \end{bmatrix}$ tiene $LU$ para cualquier $b \in \mathbb{R}^{n}$, $c^t \in \mathbb{R}^n$ y $d \in \mathbb{R}$, y hallarla.
    
    Sean
    
    \[
    \widetilde{L} =
    \begin{bmatrix}
        L_{11} & 0 \\
        l_{21}^{t} & 1
    \end{bmatrix}
    ~~~~~~~~~~~~~~~
    \widetilde{U} =
    \begin{bmatrix}
        U_{11} & u_{12} \\
        0^{t} & u_{22}
    \end{bmatrix}
    \]
    
    Se necesita que $L_{11}$ sea triangular inferior con unos en su diagonal, y $U_{11}$ sea triangular superior. Luego,
    
    \begin{itemize}
        \item $A = L_{11}U_{11}$
        \item $b = L_{11}u_{12}$
        \item $c^{t} = l_{21}^{t}U_{11}$
        \item $d = l_{21}^{t}u_{12} + u_{22}$
    \end{itemize}
    
    y como sé que $A$ es inversible y tiene factorización $LU$, tengo que $L_{11} = TD_1$ y $U = D_2 S$ . Con ello, teniendo en cuenta que $D_{1}^{-1} = D_2$,
    
    \begin{itemize}
        \item $b = L_{11}u_{12} \implies D_2 T^{-1}b = D_2 T^{-1}L_{11}u_{12} = D_2 T^{-1}TD_{1}u_{12} \implies u_{12} = D_2 T^{-1}b$
        \item $c^{t} = l_{21}^{t}U_{11} \implies c^{t}S^{-1}D_1 = l_{21}^{t}D_2 SS^{-1}D_1 \implies c^{t}S^{-1}D_1 = l_{21}^{t}$
        \item $u_{22} = d - l_{21}^{t}u_{12} = d- c^{t}S^{-1}D_1 D_2 T^{-1}b = d - c^{t}S^{-1}T^{-1}b = d - c^{t}{(TS)}^{-1}b = d - c^{t}A^{-1}b$
    \end{itemize}
    
    Luego, se puede ver que obtuve $\widetilde{L}$ triangular inferior con unos en su diagonal y $\widetilde{U}$ triangular de forma que
    
    \[
    \begin{bmatrix}
        TD_1 & 0 \\
        c^{t}S^{-1}D_1 & 1
    \end{bmatrix}
    \begin{bmatrix}
        D_2 S & D_2 T^{-1}b \\
        0^{t} & d - c^{t}A^{-1}b
    \end{bmatrix}
    = \ldots =
    \begin{bmatrix}
        A & b \\
        c^{t} & d
    \end{bmatrix}
    \]
    
    con lo que se encontró una factorización $LU$.
    
\end{itemize}
