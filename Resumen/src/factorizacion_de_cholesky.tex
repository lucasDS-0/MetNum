% factorizacion_de_cholesky

Sea $A \in \mathbb{R}^{n \times n}$, $A$ es sdp, y se tiene que

\[A = LU\]
\[A^{t} = {(LU)}^{t} = U^{t}L^{t}\]
\[A = A^t \implies LU = U^{t}L^{t}\]

como $L$ es triangular inferior (y $L^t$ triangular superior) con unos en la diagonal, es inversible

\[LU = U^t L^t \implies L^{-1}LU = L^{-1}U^{t}L^{t} \implies U = L^{-1}U^{t}L^{t} \implies 
U{(L^t)}^{-1} = L^{-1}U^{t}L^{t}{(L^t)}^{-1} \implies U{(L^t)}^{-1} = L^{-1}U\]

y tengo que $U{(L^t)}^{-1} = L^{-1}U = D$ (matriz diagonal), y entonces

\[U{(L^t)}^{-1} = D \implies U{(L^t)}^{-1}L^t = DL^t \implies U = DL^t\]

con esto se llega a que con $A \in \mathbb{R}^{n \times n}$

\[A = LU LDL^t\]

Sea $x \neq 0$ tal que $L^t x = e_i$.

\[0 < x^t Ax = x^t LDL^t x = e_{i}^{t}De_i = d_{ii}\]
\[D = \sqrt{D}\sqrt{D}\]
\[A = L\sqrt{D}\sqrt{D}L^t = \widetilde{L}\widetilde{L}^{t}\]

y $A = \widetilde{L}\widetilde{L}^{t}$ es la Factorización de Cholesky.

\subsection{Algoritmo de la Factorización de Cholesky}\label{subsec:algoritmo_de_cholesky}

El cómputo a realizar es la siguiente factorización

\[
\begin{bmatrix}
    a_{11} & a_{21} & \ldots & a_{n1} \\
    a_{21} & a_{21} & \ldots & a_{n2} \\
    \vdots & \vdots & \ddots & \vdots \\
    a_{i1} & a_{i2} & \ldots & a_{ni} \\
    \vdots & \vdots & \ddots & \vdots \\
    a_{n1} & a_{n1} & \ldots & a_{nn} \\
\end{bmatrix}
=
\begin{bmatrix}
    \tilde{l}_{11} & 0 & \ldots & 0 \\
    \tilde{l}_{21} & \tilde{l}_{21} & \ldots & 0 \\
    \vdots & \vdots & \ddots & \vdots \\
    \tilde{l}_{i1} & \tilde{l}_{i2} & \ldots & 0 \\
    \vdots & \vdots & \ddots & \vdots \\
    \tilde{l}_{n1} & \tilde{l}_{n1} & \ldots & \tilde{l}_{nn} \\
\end{bmatrix}
\begin{bmatrix}
    \tilde{l}_{11} & \tilde{l}_{21} & \ldots & \tilde{l}_{n1} \\
    0 & \tilde{l}_{21} & \ldots & \tilde{l}_{n2} \\
    \vdots & \vdots & \ddots & \vdots \\
    0 & 0 & \ldots & \tilde{l}_{ni} \\
    \vdots & \vdots & \ddots & \vdots \\
    0 & 0 & \ldots & \tilde{l}_{nn} \\
\end{bmatrix}
\]

haciendo uso del siguiente algoritmo

\begin{algorithm}
\begin{algorithmic}
    \caption{Factorización de Cholesky}\label{alg:fact_cholesky}
    \State $\tilde{l}_{11} \gets \sqrt{a_{11}}$
    \For{$j = 2\ldots n$}
        \State $\tilde{l}_{j1} \gets \frac{a_{j1}}{\tilde{l}_{11}}$
    \EndFor
    \For{$i = 2\ldots n-1$}
        \State $\tilde{l}_{ii} \gets \sqrt{a_{ii} - \sum_{k=1}^{i-1}\tilde{l}_{ik}^{2}}$
        \For{$j = i+1\ldots n$}
            \State $\tilde{l}_{ji} \gets \frac{a_{ji} - \sum_{k=1}^{i-1}\tilde{l}_{jk}\tilde{l}_{ik}}{\tilde{l}_{ii}}$
        \EndFor
    \EndFor
    \State $\tilde{l}_{nn} \gets \sqrt{a_{nn} - \sum_{k=1}^{n-1}\tilde{l}_{nk}^{2}}$
\end{algorithmic}
\end{algorithm}
    
\subsection{Ejemplo}\label{subsec:ejemplo_cholesky}

Sea $A \in \mathbb{R}^{n \times n}$  tal que $A_n = {(a_{ij})}_{1 \leq i,j \leq n}$ donde $a_{ij} = min\{i,j\}$

tengo entonces por ejemplo

\[
A_1 = 
\begin{bmatrix}
1
\end{bmatrix}
~~~~~~~~~~
A_2 =
\begin{bmatrix}
    1 & 1 \\
    1 & 2
\end{bmatrix}
~~~~~~~~~~
A_3 =
\begin{bmatrix}
    1 & 1 & 1 \\
    1 & 2 & 2 \\
    1 & 2 & 3
\end{bmatrix}
\]

\

Para probarlo hago inducción en $n$.

\begin{itemize}
    \item[-] Caso base: $n=1$, $A_1 = [1]$, con ello tengo que $x^{t}Ax = x^2 > 0$ para todo $x \neq 0$.
    \item[-] Paso inductivo: Utilizo como hipótesis inductiva el que $A_k$ es sdp $\forall~k < n$, y pruebo que $A_n$ es sdp.

    $A_n$ es simétrica, para probar que es definida positiva, busco su factorización de cholesky. Tengo que 

    \[
    \begin{bmatrix}
        1 & \vline & 1 & \ldots & 1 \\
        \hline
        1 & \vline & 2 & \ldots & 2 \\
        \vdots & \vline & \vdots & \ddots & \vdots \\
        1 & \vline & 2 & \ldots & n \\
    \end{bmatrix}
    =
    \begin{bmatrix}
        L_{11} & \vline & \overline 0^t \\
        \hline
        L_{21} & \vline & L_{22}
    \end{bmatrix}
    \begin{bmatrix}
        L_{11} & \vline & L_{21}^{t} \\
        \hline
        \overline 0 & \vline & L_{22}^{t}
    \end{bmatrix}
    =
    \begin{bmatrix}
        L_{11}^{2} & \vline & L_{11}L_{21}^{t} \\
        \hline
        L_{21}L_{11} & \vline & L_{21}L_{21}^{t} + L_{22}L_{22}^{t}
    \end{bmatrix}
    \]

    donde se puede ver que estos productos resultan ser

    \begin{itemize}
        \item[$\cdot$] $L_{11}^{2} = 1 \implies L_{11} = 1$
        \item[$\cdot$] $L_{21}L_{11} = \begin{bmatrix} 1 \\ \vdots \\ 1 \end{bmatrix} \implies 
        L_{21}= \begin{bmatrix} 1 \\ \vdots \\ 1 \end{bmatrix} \implies L_{21}^{t} = \begin{bmatrix} 1 & \ldots & 1 \end{bmatrix}$
        \item[$\cdot$] $L_{21}L_{21}^{t} + L_{22}L_{22}^{t} = 
        \begin{bmatrix} 
            2 & 2 &\ldots & 2 \\
            2 & 3 &\ldots & 3 \\
            \vdots & \vdots & \ddots & \vdots \\
            2 & 3 & \ldots & n
        \end{bmatrix}
        \implies L_{22}L_{22}^{t} = 
        \begin{bmatrix} 
            2 & 2 &\ldots & 2 \\
            2 & 3 &\ldots & 3 \\
            \vdots & \vdots & \ddots & \vdots \\
            2 & 3 & \ldots & n
        \end{bmatrix}
        -
        \begin{bmatrix} 
            1 \\
            \vdots \\
            1 
        \end{bmatrix}
        \begin{bmatrix} 
            1 & \ldots & 1
        \end{bmatrix}
        $

        lo cual es lo mismo que

        $L_{22}L_{22}^{t} = 
        \begin{bmatrix} 
            2 & 2 &\ldots & 2 \\
            2 & 3 &\ldots & 3 \\
            \vdots & \vdots & \ddots & \vdots \\
            2 & 3 & \ldots & n
        \end{bmatrix}
        -
        \begin{bmatrix} 
            1 & 1 &\ldots & 1 \\
            1 & 1 &\ldots & 1 \\
            \vdots & \vdots & \ddots & \vdots \\
            1 & 1 & \ldots & 1
        \end{bmatrix}
        =
        \begin{bmatrix} 
            1 & 1 &\ldots & 1 \\
            1 & 2 &\ldots & 2 \\
            \vdots & \vdots & \ddots & \vdots \\
            1 & 2 & \ldots & n-1
        \end{bmatrix}
        =
        A_{n-1}
        $
    \end{itemize}

    y por hipótesis inductiva, $A_{n-1}$ es sdp, lo cual implica que $A_{n-1} = \widetilde{L}\widetilde{L}^{t}$ (cholesky de $A_{n-1}$).

    Entonces $L_{22} = \widetilde{L}$ de la factorizaión de cholesky de $A_{n-1}$.

\end{itemize}

\subsection{Enunciados de la guía práctica}\label{subsec:enunciados_guia_3_cholesky}



\subsubsection{Ejercicio 11}\label{subsubsec:guia_3_ej_11}