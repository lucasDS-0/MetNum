\documentclass{article}
\usepackage[utf8]{inputenc}
\usepackage{algorithmicx, algpseudocode, algorithm}
\usepackage{tikz}
\usepackage{amsmath}
\usepackage{amsthm}
\usepackage{multicol}
\usepackage{amssymb}
\usepackage[paper=a4paper, left=1.5cm, right=1.5cm, bottom=1.5cm, top=1.5cm]{geometry}
\usepackage{xcolor}

\title{}
\author{}
\date{}

\begin{document}

\section*{Discretización}
Planteo una dsicretización con  $n = 3$ y $m = 3$

\section*{Matriz}

Luego las temperaturas del horno son:

\begin{center}
    \begin{tabular}{|c|c|c|c|}
        \hline
        $t_{00}$ & $t_{10}$ & $t_{20}$ & $t_{30}$ \\
        \hline
        $t_{01}$ & $t_{11}$ & $t_{21}$ & $t_{31}$ \\
        \hline
        $t_{02}$ & $t_{12}$ & $t_{22}$ & $t_{32}$ \\
        \hline
        $t_{03}$ & $t_{13}$ & $t_{23}$ & $t_{33}$ \\
        \hline
    \end{tabular}
\end{center}

\

\begin{itemize}
    \item[-] Por enunciado tengo que $t_{00} = t_{01} = t_{02} = t_{03} = 1500$
    \item[-] Adicionalmente, como $\theta_0 = \theta_{n}$, tengo que $t_{00} = t_{03},~t_{10} = t_{13},~t_{20} = t_{23}$ y 
          $t_{30} = t_{33}$
    \item[-] Por último, elijo la temperatura para la pared exterior, $t_{30} = t_{31} = t_{32} = t_{33} = 200$
    \item[-] Con esto las incógnitas a calcular que tengo son: $t_{11},~t_{12},~t_{21},~t_{22},~t_{13},~t_{33}$
\end{itemize}

\begin{center}
    \begin{tabular}{|c|c|c|c|}
        \hline
        $1500$ & $t_{11}$ & $t_{21}$ & $200$ \\
        \hline
        $1500$ & $t_{12}$ & $t_{22}$ & $200$ \\
        \hline
        $1500$ & $t_{13}$ & $t_{23}$ & $200$ \\
        \hline
    \end{tabular}
\end{center}

\section*{Ecuación}

La ecuación de calor discretizada para cada $t_{jk}$ es la siguiente:

\begin{align*}
    T(r_j,\theta_k) =-\frac{[{(\Delta r)}^{-2} - {(r_j\Delta r)}^{-1}]}{[{(r_j\Delta r)}^{-1} - 2{(\Delta r)}^{-2} - 2{(r_{j}\Delta \theta)}^{-2}]}t_{j-1,k}
    & \\
    -\frac{[{(r_{j}\Delta \theta)}^{-2}]}{[{(r_j\Delta r)}^{-1} - 2{(\Delta r)}^{-2} - 2{(r_{j}\Delta \theta)}^{-2}]}(t_{j,k-1} + t_{j,k+1})
    -\frac{[{(\Delta r)}^{-2}]}{[{(r_j\Delta r)}^{-1} - 2{(\Delta r)}^{-2} - 2{(r_{j}\Delta \theta)}^{-2}]}t_{j+1,k}  - t_{jk} &= 0
\end{align*}

\

\noindent Por comodidad, nombro a cada coeficiente de la siguiente forma:

\begin{align*}
    &-\frac{[{(\Delta r)}^{-2} - {(r_j\Delta r)}^{-1}]}{[{(r_j\Delta r)}^{-1} - 2{(\Delta r)}^{-2} - 2{(r_{j}\Delta \theta)}^{-2}]}t_{j-1,k}& &= -a\cdot t_{j-1,k} \\
    &-\frac{[{(r_{j}\Delta \theta)}^{-2}]}{[{(r_j\Delta r)}^{-1} - 2{(\Delta r)}^{-2} - 2{(r_{j}\Delta \theta)}^{-2}]}(t_{j,k-1} + t_{j,k+1})& &= -b\cdot (t_{j,k-1} + t_{j,k+1}) \\
    &-\frac{[{(\Delta r)}^{-2}]}{[{(r_j\Delta r)}^{-1} - 2{(\Delta r)}^{-2} - 2{(r_{j}\Delta \theta)}^{-2}]}t_{j+1,k}& &= -c\cdot t_{j+1,k}
\end{align*}

\noindent Con esto ya se puede plantear el sistema de ecuacines para buscar el valor de las incógnitas.

\newpage

\section*{Sistema}

\emph{Nota: Los valores en rojo son aquellos que se encuentran en $r_i$ o $r_e$, por ende al saberse su valor son lost terminos independientes de cada ecuación}

\begin{align*}
    &\textcolor{red}{-a\cdot t_{01}}& &-b\cdot(t_{13} +t_{12})& &-c\cdot t_{21}& &-t_{11}& &= 0 \\
    &\textcolor{red}{-a\cdot t_{02}}& &-b\cdot(t_{11} +t_{13})& &-c\cdot t_{22}& &-t_{12}& &= 0 \\
    &\textcolor{red}{-a\cdot t_{03}}& &-b\cdot(t_{12} +t_{11})& &-c\cdot t_{23}& &-t_{13}& &= 0 \\
    &-a\cdot t_{11}& &-b\cdot(t_{23} +t_{22})& &~\textcolor{red}{-c\cdot t_{31}}& &-t_{21}& &= 0 \\
    &-a\cdot t_{12}& &-b\cdot(t_{21} +t_{23})& &~\textcolor{red}{-c\cdot t_{32}}& &-t_{22}& &= 0 \\
    &-a\cdot t_{13}& &-b\cdot(t_{22} +t_{21})& &~\textcolor{red}{-c\cdot t_{33}}& &-t_{23}& &= 0 \\
\end{align*}

\section*{Matriz banda}

Reacomodando el sistema me encuentro con que puedo representar la información como una matriz banda:

\

\begin{center}
    \begin{tabular}{c c c c c c c c}
        $-t_{11}$ & $-b\cdot t_{12}$ & $-b\cdot t_{13}$ & $-c\cdot t_{21}$ & $0$ & $0$ & $=$ & $a\cdot t_{01}$ \\
        $-b\cdot t_{11}$ & $-t_{12}$ & $-b\cdot t_{13}$ & $0$ & $-c\cdot t_{22}$ & $0$ & $=$ & $a\cdot t_{02}$ \\
        $-b\cdot t_{11}$ & $-b\cdot t_{12}$ & $-t_{13}$ & $0$ & $0$ & $-c\cdot t_{33}$ & $=$ & $a\cdot t_{03}$ \\
        $-a\cdot t_{11}$ & $0$ & $0$ & $-t_{21}$ & $-b\cdot t_{22}$ & $-b\cdot t_{23}$ & $=$ & $a\cdot t_{31}$ \\
        $0$ & $-a\cdot t_{12}$ & $0$ & $-b\cdot t_{21}$ & $-t_{22}$ & $-b\cdot t_{23}$ & $=$ & $a\cdot t_{32}$ \\
        $0$ & $0$ & $-a\cdot t_{13}$ & $-b\cdot t_{21}$ & $-b\cdot t_{22}$ & $-t_{23}$ & $=$ & $a\cdot t_{33}$
    \end{tabular}
\end{center}

\

de esta manera, la matriz con la que se deberá trabajar es la siguiente:

\

\[
\begin{bmatrix}
    -1 & -b & -b & -c & 0 & 0 & | & a \\
    -b & -1 & -b & 0 & -c & 0 & | & a \\
    -b & -b & -1 & 0 & 0 & -c & | & a \\
    -a & 0 & 0 & -1 & -b & -b & | & c \\
    0 & -a & 0 & -b & -1 & -b & | & c \\
    0 & 0 & -a & -b & -b & -1 & | & c
\end{bmatrix}
\]

\end{document}
