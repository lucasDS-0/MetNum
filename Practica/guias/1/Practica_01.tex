\documentclass{article}
\usepackage[utf8]{inputenc}
\usepackage{algorithmicx, algpseudocode, algorithm}
\usepackage{tikz}
\usepackage{amsmath}
\usepackage{amsthm}
\usepackage{multicol}
\usepackage{amssymb}
\usepackage[paper=a4paper, left=1.5cm, right=1.5cm, bottom=1.5cm, top=1.5cm]{geometry}

\title{Práctica: Elementos de Álgebra Lineal}
\author{Lucas Di Salvo}
\date{}

\begin{document}

\maketitle

\emph{Nota: $\mathbb{R}^{n}$ está formado por vectores columna. 
    Cuando se escriben por filas es por comodidad tipográfica.}

\section*{Ejercicios}

\subsection*{Ejercicio 1}

Información:

\begin{multicols}{3}
    \begin{itemize}
        \item[] $A = (a_{ij}) \in \mathbb{R}^{n\times m}$
        \item[] $B = (b_{ij}) \in \mathbb{R}^{m\times n}$
        \item[] $D = (d_{ij}) \in \mathbb{R}^{m\times m}$
        \item[] $x = (x_i) \in R^n$
        \item[] $z = (z_i) \in R^n$
        \item[] $y = (y_i) \in R^m$
        \item[] $w = (w_i) \in R^m$
    \end{itemize}
\end{multicols}


\subsubsection*{a}
Falsa, pues la cantidad de elementos por fila de $A$ no es igual a la cantidad de elementos por fila de $z$.
\subsubsection*{b}
Verdadera.
\subsubsection*{c}
Verdadera.
\subsubsection*{d}
$d^{-1}$???
\subsection*{Ejercicio 2}

\subsubsection*{a}

\subsubsection*{b}

\subsubsection*{c}


\subsection*{Ejercicio 3}

\subsubsection*{a}

\subsubsection*{b}

\subsection*{Ejercicio 4}

\subsection*{Ejercicio 5}

\subsection*{Ejercicio 6}

\subsection*{Ejercicio 7}

\subsection*{Ejercicio 8}

\subsection*{Ejercicio 9}

\subsubsection*{a}

\subsubsection*{b}

\subsection*{Ejercicio 10}

\subsubsection*{a}

\subsubsection*{b}

\subsection*{Ejercicio 11}

\subsubsection*{a}

\subsubsection*{b}

\subsection*{Ejercicio 12}

\subsection*{Ejercicio 13}

\subsection*{Ejercicio 14}

\subsection*{Ejercicio 15}

\subsection*{Ejercicio 16}

\subsection*{Ejercicio 17}

\subsection*{Ejercicio 18}

\subsubsection*{a}

\subsubsection*{b}

\subsection*{Ejercicio 19}

\subsubsection*{a}

\subsubsection*{b}

\subsubsection*{c}

\subsection*{Ejercicio 20}

\subsubsection*{a}

\subsubsection*{b}

\subsubsection*{c}

\subsection*{Ejercicio 21}

\subsubsection*{a}

\subsubsection*{b}

\subsubsection*{c}

\subsubsection*{d}

\subsection*{Ejercicio 22}

\subsubsection*{a}

\subsubsection*{b}

\subsubsection*{c}

\subsubsection*{d}

\subsection*{Ejercicio 23}

\subsubsection*{a}

\subsubsection*{b}

\subsubsection*{c}

\subsection*{Ejercicio 24}

\subsubsection*{a}

\subsubsection*{b}

\subsection*{Ejercicio 25}

\subsubsection*{a}

\subsubsection*{b}

\subsection*{Ejercicio 26}

\subsection*{Ejercicio 27}

\subsubsection*{a}

\subsubsection*{b}

\subsubsection*{c}

\subsection*{Ejercicio 28}

\subsubsection*{a}

\subsubsection*{b}

\subsection*{Ejercicio 29}

\subsection*{Ejercicio 30}

\section*{Resolver en computadora}

\subsection*{Ejercicio I}

\subsection*{Ejercicio II}

\subsubsection*{a}

\subsubsection*{b}

\subsubsection*{c}

\subsection*{Ejercicio III}

\subsubsection*{a}

\subsubsection*{b}

\section*{Referencias}

\end{document}