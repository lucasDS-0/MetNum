\documentclass{article}
\usepackage[utf8]{inputenc}
\usepackage{algorithmicx, algpseudocode, algorithm}
\usepackage{tikz}
\usepackage{amsmath}
\usepackage{amsthm}
\usepackage{multicol}
\usepackage{amssymb}
\usepackage[paper=a4paper, left=1.5cm, right=1.5cm, bottom=1.5cm, top=1.5cm]{geometry}

\title{Práctica 1: Elementos de Álgebra Lineal}
\author{Lucas Di Salvo}
\date{}

\begin{document}

\maketitle

\emph{Nota: $\mathbb{R}^{n}$ está formado por vectores columna. 
    Cuando se escriben por filas es por comodidad tipográfica.}

\section*{Ejercicios}

\subsection*{Ejercicio 1}

Información:

\begin{multicols}{3}
    \begin{itemize}
        \item[] $A = (a_{ij}) \in \mathbb{R}^{n\times m}$
        \item[] $B = (b_{ij}) \in \mathbb{R}^{m\times n}$
        \item[] $D = (d_{ij}) \in \mathbb{R}^{m\times m}$
        \item[] $x = (x_i) \in R^n$
        \item[] $z = (z_i) \in R^n$
        \item[] $y = (y_i) \in R^m$
        \item[] $w = (w_i) \in R^m$
    \end{itemize}
\end{multicols}


\subsubsection*{a}
Falsa, pues la cantidad de elementos por fila de $A$ no es igual a la cantidad de elementos por fila de $z$.

\begin{multicols}{3}
\subsubsection*{b}
Verdadera.
\subsubsection*{c}
Verdadera.
\subsubsection*{d}
$d^{-1}$???
\end{multicols}
    \subsection*{Ejercicio 2}

Información:

\

\begin{center}
$A = \begin{pmatrix}
    1 & 3 & 0 \\
    0 & 1 & 2 \\
    1 & 0 & 1
\end{pmatrix} 
~~~~~~~~~~~~~~~~~~
B = \begin{pmatrix}
    1 & 1 & 1 \\
    3 & 0 & 1 \\
    2 & 0 & 2
\end{pmatrix}
~~~~~~~~~~~~~~~~~~
C = \begin{pmatrix}
    c_{11} & c_{12} & c_{13} \\
    c_{21} & c_{22} & c_{23} \\
    c_{31} & c_{32} & c_{33}
\end{pmatrix}$
\end{center}

\subsubsection*{a}
Es realizable.

\

$C_{11} = A_{11}B_{11} + A_{12}B_{21} =  
\begin{bmatrix}
    1    
\end{bmatrix}
\cdot
\begin{bmatrix}
    1
\end{bmatrix}
+
\begin{bmatrix}
    3 & 0
\end{bmatrix}
\cdot
\begin{bmatrix}
    3 \\
    2
\end{bmatrix} 
=
\begin{bmatrix}
    10
\end{bmatrix} \in \mathbb{Z}^{1\times 1}$

\

$C_{12} = A_{11}B_{12} + A_{12}B_{22} =  
\begin{bmatrix}
    1    
\end{bmatrix}
\cdot
\begin{bmatrix}
    1 & 1
\end{bmatrix}
+
\begin{bmatrix}
    3 & 0
\end{bmatrix}
\cdot
\begin{bmatrix}
    0 & 1 \\
    0 & 2
\end{bmatrix}
=
\begin{bmatrix}
    1 & 4
\end{bmatrix} \in \mathbb{Z}^{1\times 2}$

\

$C_{21} = A_{21}B_{11} + A_{22}B_{21} =  
\begin{bmatrix}
    0 \\
    1    
\end{bmatrix}
\cdot
\begin{bmatrix}
    1
\end{bmatrix}
+
\begin{bmatrix}
    1 & 2 \\
    0 & 1
\end{bmatrix}
\cdot
\begin{bmatrix}
    3 \\
    2
\end{bmatrix}
=
\begin{bmatrix}
    7 \\
    3
\end{bmatrix} \in \mathbb{Z}^{2\times 1}$

\

$C_{22} = A_{21}B_{12} + A_{22}B_{22} =  
\begin{bmatrix}
    0 \\
    1    
\end{bmatrix}
\cdot
\begin{bmatrix}
    1 & 1
\end{bmatrix}
+
\begin{bmatrix}
    1 & 2 \\
    0 & 1
\end{bmatrix}
\cdot
\begin{bmatrix}
    0 & 1 \\
    0 & 2
\end{bmatrix}
=
\begin{bmatrix}
    0 & 5 \\
    1 & 3
\end{bmatrix} \in \mathbb{Z}^{2\times 2}$

\subsubsection*{b}

No es realizable.

\subsubsection*{c}

Es realizable.

\

$C_{11} = A_{11}B_{11} + A_{12}B_{21} =  
\begin{bmatrix}
    1 \\
    0
\end{bmatrix}
\cdot
\begin{bmatrix}
    1
\end{bmatrix}
+
\begin{bmatrix}
    3 & 0 \\
    1 & 2
\end{bmatrix}
\cdot
\begin{bmatrix}
    3 \\
    2
\end{bmatrix} 
=
\begin{bmatrix}
    10 \\
    7
\end{bmatrix} \in \mathbb{Z}^{2\times 1}$

\

$C_{12} = A_{11}B_{12} + A_{12}B_{22} =  
\begin{bmatrix}
    1 \\
    0    
\end{bmatrix}
\cdot
\begin{bmatrix}
    1 & 1
\end{bmatrix}
+
\begin{bmatrix}
    3 & 0 \\
    1 & 2
\end{bmatrix}
\cdot
\begin{bmatrix}
    0 & 1 \\
    0 & 2
\end{bmatrix}
=
\begin{bmatrix}
    1 & 4 \\
    0 & 5
\end{bmatrix} \in \mathbb{Z}^{2\times 2}$

\

$C_{21} = A_{21}B_{11} + A_{22}B_{21} =  
\begin{bmatrix}
    1    
\end{bmatrix}
\cdot
\begin{bmatrix}
    1
\end{bmatrix}
+
\begin{bmatrix}
    0 & 1
\end{bmatrix}
\cdot
\begin{bmatrix}
    3 \\
    2
\end{bmatrix}
=
\begin{bmatrix}
    3
\end{bmatrix} \in \mathbb{Z}^{1\times 1}$

\

$C_{22} = A_{21}B_{12} + A_{22}B_{22} =  
\begin{bmatrix}
    1
\end{bmatrix}
\cdot
\begin{bmatrix}
    1 & 1
\end{bmatrix}
+
\begin{bmatrix}
    0 & 1
\end{bmatrix}
\cdot
\begin{bmatrix}
    0 & 1 \\
    0 & 2
\end{bmatrix}
=
\begin{bmatrix}
    1 & 3
\end{bmatrix} \in \mathbb{Z}^{1\times 2}$

\

\noindent ¿Qué otras particiones válidas son posibles?

\

\noindent Hay dos más.

\subsection*{Ejercicio 3}


\textcolor{red}{falta pensar}

\subsubsection*{a}



\subsubsection*{b}

\subsection*{Ejercicio 4}

$n = 2, ~
A = \begin{bmatrix}
    1 & 1 \\
    0 & 1    
\end{bmatrix}, ~
B = \begin{bmatrix}
    0 & 1 \\
    1 & 0
\end{bmatrix}$ pues 
$AB = \begin{bmatrix}
    1 & 1 \\
    1 & 0
\end{bmatrix}$ y 
$BA = \begin{bmatrix}
    0 & 1 \\
    1 & 1
\end{bmatrix}$, donde $tr(AB) = 1$ y $tr(A)tr(B) = 0$


\subsection*{Ejercicio 5}

Falso, por ejemplo:

\

\noindent $m = n = r = 2, ~
A = \begin{bmatrix}
    1 & 0 \\
    0 & 0
\end{bmatrix}, ~
B = \begin{bmatrix}
    0 & 0 \\
    0 & 1
\end{bmatrix}$, donde se tiene que $AB = 0$, pero $A \neq 0$ y $B \neq 0$.

\subsection*{Ejercicio 6}

Falso, por ejemplo:

\

\noindent $m = n = 2, ~
A = \begin{bmatrix}
    1 & 0 \\
    0 & 0
\end{bmatrix}, ~
B = \begin{bmatrix}
    0 & 0 \\
    0 & 1
\end{bmatrix}, ~
C = \begin{bmatrix}
    0 & 0 \\
    0 & 0
\end{bmatrix}$ donde se tiene que $A$ no nula, $AB = AC$ y $B \neq C$.

\subsection*{Ejercicio 7}

Sean $A, B \in \mathbb{R}^{n\times n}$. 

\noindent Tengo que dar las condiciones necesarias para que

\[{(A+B)}^{2} = A^2 + 2AB + B^2\]

observo que 

\[{(A+B)}^{2} = (A+B)(A+B) = A^{2} + AB + BA + B^{2}\]

\noindent Con lo que para que se cumpla la igualdad pedida requiero que $AB = BA$, es decir que $A$ y $B$

\textcolor{red}{falta completar}

%TODO falta completar

\

\noindent Por otro lado tengo que dar las condiciones necesarias para que 

\[(A+B)(A-B) = A^2 - B^2\]

observo que 

\[(A+B)(A-B) = A^2 - AB + BA - B^2\]

\noindent Con lo que para que se cumpla la igualdad pedida requiero que $-AB = BA$, es decir que $A$ y $B$

\textcolor{red}{falta completar}

%TODO falta completar


\subsection*{Ejercicio 8}

\textcolor{red}{falta pensar}

\subsection*{Ejercicio 9}

\subsubsection*{a}

Es L.I.

\subsubsection*{b}

No es L.I., $1\cdot(3,3,3) + 2\cdot(2,1,0) = (7,5,3)$.

\subsection*{Ejercicio 10}

\subsubsection*{a}

Dado que los vectores no son L.I. (pues $-4\cdot(1,2,0)+ 5\cdot(1,3,6) = (1,7,30)$), 
tomo las siguientes dos bases y las extiendo con el vector canonico correspondiente.
\begin{itemize}
    \item $\langle (1,2,0), (1,3,6), (0,1,0) \rangle$
    \item $\langle (1,2,0), (1,7,30), (1,0,0) \rangle$
\end{itemize}

\subsubsection*{b}

Dado que los vectores no son L.I. (pues $4\cdot(1,2) = (4,8)$), 
tomo las siguientes dos bases y las extiendo con el vector canonico correspondiente.
\begin{itemize}
    \item $\langle (1,2), (0,1) \rangle$
    \item $\langle (1,2), (1,0) \rangle$
\end{itemize}

\subsection*{Ejercicio 11}

\subsubsection*{a}

Sea $\lambda \in \mathbb{R},~ \lambda \neq 0$. Quiero probar que:

\begin{center}
    $\{v_1,\ldots,v_i,\ldots,v_m\} \subseteq \mathbb{R}^{n}$ con $m \leq n$ es L.I. 
    $\iff$ 
    $\{v_1,\ldots,\lambda v_i,\ldots,v_m\}$ es L.I.
\end{center}

\noindent procedo a probar ambas implicaciones

\begin{itemize}
    \item[] $(\implies)$ Tengo que $\{v_1,\ldots,v_i,\ldots,v_m\}$ es L.I., lo cual significa que para una combinación 
    lineal \\ $w = \sum_{i=1}^{m} \alpha_{i}v_{i}, \text{si} ~w = 0 \implies \alpha_i = 0 ~ \forall i = 1,\ldots,m$. \\ En
    particular, esto aplica para cualquier producto por escalar $v_j = \lambda v_i$ (con $\lambda \in \mathbb{R}, ~ \lambda \neq 0$ y $1 \leq j \leq m$),
    pues al tratarse de vectores linealmente independientes se tiene que \\
    $0 = \alpha_1v_1+\cdots+\alpha_{i}v_{i}+\cdots+\alpha_{m}v_{m} = 
    \alpha_1v_1+\cdots+\alpha_{i}v_{j}+\cdots+\alpha_{m}v_{m} = 
    \alpha_1v_1+\cdots+\alpha_{i}\lambda v_{i}+\cdots+\alpha_{m}v_{m} = 0$, \\
    con lo que $\{v_1,\ldots,\lambda v_i,\ldots,v_m\}$ es L.I.
    \item[] $(\impliedby)$ Tengo que $\{v_1,\ldots,\lambda v_i,\ldots,v_m\}$ es L.I., lo cual significa que para una combinación 
    lineal \\ $w = \sum_{j=1 / j \neq i}^{m} \alpha_{j}v_{j} + \alpha_i\lambda v_i, \text{si} ~w = 0 \implies \alpha_j = 0 ~ \forall i = 1,\ldots,m$. \\
    En particular, esto aplica para cualquier producto por escalar $v_j = \frac{1}{\lambda}v_i$ (con $\lambda \in \mathbb{R}, ~\lambda \neq 0$ y $1 \leq j \leq m$),
    pues al tratarse de vectores linealmente independientes se tiene que \\
    $0 = \alpha_1v_1+\cdots+\alpha_{i}\lambda v_{i}+\cdots+\alpha_{m}v_{m} = 
    \alpha_1v_1+\cdots+\alpha_{i}v_{j}+\cdots+\alpha_{m}v_{m} = 
    \alpha_1v_1+\cdots+\alpha_{i}v_{i}+\cdots+\alpha_{m}v_{m} = 0$, \\
    con lo que $\{v_1,\ldots,v_i,\ldots,v_m\}$ es L.I.
\end{itemize}

\noindent Con esto se prueba lo pedido.

\subsubsection*{b}

Sea $\lambda \in \mathbb{R}$. Quiero probar que:

\begin{center}
    $\{v_1,\ldots,v_i,\ldots,v_j,\ldots,v_m\} \subseteq \mathbb{R}^{n}$ con $m \leq n$ es L.I. 
    $\iff$ 
    $\{v_1,\ldots,v_i + \lambda v_j,\ldots,v_j,\ldots,v_m\}$ es L.I.
\end{center}

\begin{itemize}
    \item[] $(\implies)$ \textcolor{red}{usar el punto (a)}
    \item[] $(\impliedby)$ 
\end{itemize}

\subsection*{Ejercicio 12}

\

\subsection*{Ejercicio 13}

\

\subsection*{Ejercicio 14}

\

\subsection*{Ejercicio 15}

\[f(x_1,x_2,x_3) = (x_1 + 2x_2 + 3x_3, 7x_1 + 4x_2 + 3x_3)\]

\subsection*{Ejercicio 16}

\[A = \begin{bmatrix}
    1 & 2 & 3 \\
    2 & 0 & 3 \\
    0 & 3 & 2
\end{bmatrix}\]

\newpage

\subsection*{Ejercicio 17}

Para \[A = \begin{pmatrix}
    1 & 2 & 3 \\
    2 & 1 & 2 \\
    1 & 3 & 4 \\
    0 & 0 & 1
\end{pmatrix}\]

\noindent Procedo a triangular $A$:

\[\begin{bmatrix}
    1 & 2 & 3 \\
    2 & 1 & 2 \\
    1 & 3 & 4 \\
    0 & 0 & 1
\end{bmatrix}
\underset{F_3-F_1}{\underset{F_2 - 2F_1}{\rightarrow}}
\begin{bmatrix}
    1 & 2 & 3 \\
    0 & -3 & -4 \\
    0 & 1 & 1 \\
    0 & 0 & 1
\end{bmatrix}
F_2 \leftrightarrow F_3
\begin{bmatrix}
    1 & 2 & 3 \\
    0 & 1 & 1 \\
    0 & -3 & -4 \\
    0 & 0 & 1
\end{bmatrix}
\underset{F_3 - (-3)F_2}{\rightarrow}
\begin{bmatrix}
    1 & 2 & 3 \\
    0 & 1 & 1 \\
    0 & 0 & -1 \\
    0 & 0 & 1
\end{bmatrix}
\underset{F_4 - (-1)F_3}{\rightarrow}
\begin{bmatrix}
    1 & 2 & 3 \\
    0 & 1 & 1 \\
    0 & 0 & -1 \\
    0 & 0 & 0
\end{bmatrix}
\]

\

\noindent Luego tomo las 3 primeras filas ya que son L.I. y formulo una combinación lineal con parametros 
de los 3 vectores que generan el subespacio:

\

\[v =  
\alpha\begin{bmatrix}
    1 \\
    2 \\
    3
\end{bmatrix}
+
\beta\begin{bmatrix}
    0 \\ 
    1 \\
    1
\end{bmatrix}
+
\delta\begin{bmatrix}
    0 \\
    0 \\
    -1
\end{bmatrix}
\]

\

\noindent Con esto puedo calcular la imagen multiplicando cada uno por la matriz original para encontrar una base de la imagen.

\

\[Av = 
\alpha\begin{bmatrix}
    1 & 2 & 3 \\
    2 & 1 & 2 \\
    1 & 3 & 4 \\
    0 & 0 & 1
\end{bmatrix}\cdot
\begin{bmatrix}
    1 \\
    2 \\
    3
\end{bmatrix}
+
\beta\begin{bmatrix}
    1 & 2 & 3 \\
    2 & 1 & 2 \\
    1 & 3 & 4 \\
    0 & 0 & 1
\end{bmatrix}\cdot
\begin{bmatrix}
    0 \\
    1 \\
    1
\end{bmatrix}
+
\delta\begin{bmatrix}
    1 & 2 & 3 \\
    2 & 1 & 2 \\
    1 & 3 & 4 \\
    0 & 0 & 1
\end{bmatrix}\cdot
\begin{bmatrix}
    0 \\
    0 \\
    -1
\end{bmatrix}
=
\alpha\begin{bmatrix}
    14 \\
    10 \\
    19 \\
    3 
\end{bmatrix}
+
\beta\begin{bmatrix}
    5 \\
    3 \\
    7 \\
    1 
\end{bmatrix}
+
\delta\begin{bmatrix}
    -3 \\
    -2 \\
    -4 \\
    -1
\end{bmatrix}
\]

\

\noindent Y entonces obtengo $Im(A) = \langle (14,10,19,3),(5,3,7,1),(-3,-2,-4,-1)\rangle$

\

\noindent Para el nucleo de $A$ tengo que triangular su matriz extendida con el vector nulo:

\

\[
\begin{bmatrix}
    1 & 2 & 3 & | & 0 \\
    2 & 1 & 2 & | & 0 \\
    1 & 3 & 4 & | & 0 \\
    0 & 0 & 1 & | & 0
\end{bmatrix}
\sim
\begin{bmatrix}
    1 & 2 & 3 & | & 0 \\
    0 & 1 & 1 & | & 0 \\
    0 & 0 & -1 & | & 0 \\
    0 & 0 & 0 & | & 0
\end{bmatrix}
\]

\

\noindent Dado que todo se iguala a cero, me encuentro con que $Nu(A) = \{0\}$

\

\noindent Por otro lado, como luego de triangular $A$, solo 3 columnas y filas resultaron ser L.I.,
su rango fila y su rango columna es 3.

\

Con ello se concluye:

\begin{itemize}
    \item $Nu(A) = \{0\}$ 
    \item $Im(A) = \langle (14,10,19,3),(5,3,7,1),(-3,-2,-4,-1)\rangle$ 
    \item $rg(A) = 3$
    \item $n = \dim(Nu(A)) + \dim(Im(A)) = 3$  
\end{itemize}

\newpage

\noindent Para \[A = \begin{pmatrix}
    1 & 1 & 1 \\
    2 & 3 & 7 \\
    1 & 6 & 30
\end{pmatrix}\]

\noindent se tiene 

\begin{itemize}
    \item $Nu(A) = $ 
    \item $Im(A) = $ 
    \item $rg(A) = $
    \item $e = \dim(Nu(A)) + \dim(Im(A)) = $  
\end{itemize}

\subsection*{Ejercicio 18}

\

\subsubsection*{a}

\subsubsection*{b}

\subsection*{Ejercicio 19}

\

\subsubsection*{a}

\subsubsection*{b}

\subsubsection*{c}

\subsection*{Ejercicio 20}

\

\subsubsection*{a}

\subsubsection*{b}

\subsubsection*{c}

\subsection*{Ejercicio 21}

\

\subsubsection*{a}

\subsubsection*{b}

\subsubsection*{c}

\subsubsection*{d}

\subsection*{Ejercicio 22}

\

\subsubsection*{a}

\subsubsection*{b}

\subsubsection*{c}

\subsubsection*{d}

\subsection*{Ejercicio 23}

\

\subsubsection*{a}

\subsubsection*{b}

\subsubsection*{c}

\subsection*{Ejercicio 24}

\

\subsubsection*{a}

\subsubsection*{b}

\subsection*{Ejercicio 25}

\

\subsubsection*{a}

\subsubsection*{b}

\subsection*{Ejercicio 26}

\

\subsection*{Ejercicio 27}

\

\subsubsection*{a}

\subsubsection*{b}

\subsubsection*{c}

\subsection*{Ejercicio 28}

\

\subsubsection*{a}

\subsubsection*{b}

\subsection*{Ejercicio 29}

\

\subsection*{Ejercicio 30}

\

\section*{Resolver en computadora}

\subsection*{Ejercicio I}

\

\subsection*{Ejercicio II}

\

\subsubsection*{a}

\subsubsection*{b}

\subsubsection*{c}

\subsection*{Ejercicio III}

\

\subsubsection*{a}

\subsubsection*{b}

\section*{Referencias}

\

\end{document}